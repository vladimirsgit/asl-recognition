\begin{abstractpage}

\begin{abstract}{romanian}

Scopul acestei lucrări este facilitarea interacțiunii cu persoanele care prezintă deficiențe de auz și/sau vorbire prin eliminarea barierei de comunicare prezente între acestea și indivizii din populația generală.

Soluția propusă constă într-o aplicație mobilă capabilă să detecteze și să traducă semnele alfabetului \textit{American Sign Language} (ASL). Nucleul aplicației este reprezentat de o rețea neuronală convoluțională, integrată într-o aplicație dezvoltată pentru sistemul de operare Android. Modelul urmează o arhitectură proprie, inspirată din modele performante existente, și este antrenat pe un set de date construit din multiple surse. 

În procesul de antrenare au fost utilizate tehnici moderne de regularizare, precum \textit{dropout}, \textit{label smoothing} și augmentarea datelor, cu scopul îmbunătățirii generalizării modelului. Acuratețea finală pe un set de testare reprezentativ pentru mediul real este de peste $91\%$.

Se poate concluziona că lucrarea și-a atins obiectivul, modelul propus fiind performant și eficient, iar aplicația oferă o bază pentru extinderea cercetării către recunoașterea cuvintelor și a propozițiilor.
 
\end{abstract}

\begin{abstract}{english}

The aim of this thesis is to bridge the gap between individuals with hearing and/or speech impairments and the general population.

The proposed solution consists of a mobile application capable of detecting and translating the American Sign Language (ASL) alphabet. At its core, the application is composed of a convolutional neural network, integrated into an Android mobile application. The model follows a custom architecture, inspired by state-of-the-art models, and was trained on a custom dataset composed of multiple sources.

During the training process, several modern regularization techniques were applied, such as dropout, label smoothing and data augmentation, with the purpose of improving model generalization. The final model achieved an accuracy of over $91\%$ on a test set representative of real-world conditions.

It can be concluded that the thesis reached its goal, the proposed model being accurate and efficient. Moreover, the final application offers a foundation for extending the research towards recognizing words and sentences.

\end{abstract}

\end{abstractpage}